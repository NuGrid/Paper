\documentclass[onecolumn]{aastex3}

\newcommand\aastex{AAS\TeX}%

\begin{document}

\begin{center}
Using {\tt\string natbib} with \aastex
\end{center}

\aastex\ uses Patrick Daly's {\tt\string natbib} package to manage
citations. The {\tt\string natbib} package re-implements LaTeX's
{\tt\string\cite} command, and offers greater flexibility for managing
citations in the author-year form. {\tt\string natbib} is required with the
\aastex\ package but fortunately {\tt\string natbib} is included in most
modern distributions. Authors are encouraged to read the {\tt\string
natbib} documentation for complete details on the package's capabilites.
All examples in this document are adapted from the {\tt\string natbib}
documentation.

Authors must use the LaTeX's {\tt\string thebibliography} environment to
use the {\tt\string natbib} extensions. Here is a bibliography with two
entries marked up in the {\tt\string natbib} style:

\begin{verbatim}
\begin{thebibliography}{}
\bibitem[James et al.(1991)]{jam91} James, H. ...
\bibitem[Jones et al.(1990)Jones, Baker, and Williams]{jon90} Jones, J. ...
\end{thebibliography}
\end{verbatim}

The square-bracketed argument contains the "author" portion of the citation
followed by the year in parentheses. The parentheses are important, so do
not leave them out. Note that in the second {\tt\string\bibitem}, the
square-bracketed argument includes two author lists, a short version before
the year and a long version after. Authors should use this form of the
markup if they wish to use the * forms of the {\tt\string\cite} commands.
(See the examples below.)

The text in the curly-brace argument of the {\tt\string\bibitem} is the
citation key and should be used as the argument in the corresponding cite
commands in the text. The two basic text citation commands are
{\tt\string\citet} and {\tt\string\citep}. ({\tt\string\citet} corresponds
to the plain LaTeX {\tt\string\cite} command.) Use the two optional
arguments to append text before or after the citationtext in the first set
of square brackets will appear before the cite, text in the second set will
appear after it. Use the asterisk (*) form of the cite commands to print
the long version of the author lists for references that have been marked
up in the manner of the "jon90" example.

Here are some {\tt\string\citet} and {\tt\string\citep} examples taken from
the {\tt\string natbib} package
documentation\footnote{\url{http://www.ctan.org/pkg/natbib}}.

\begin{tabular}{ll}
\textbf{Command} & \textbf{Output} \\
\verb|\citet{jon90}| & Jones et al. (1990) \\
\verb|\citet[chap.~2]{jon90}| & Jones et al. (1990, chap.~2) \\
\verb|\citep{jon90}| & (Jones et al., 1990) \\
\verb|\citep[chap.~2]{jon90}| & (Jones et al., 1990, chap.~2) \\
\verb|\citep[see][]{jon90}| & (see Jones et al., 1990) \\
\verb|\citep[see][chap.~2]{jon90}| & (see Jones et al., 1990, chap.~2) \\
\verb|\citet*{jon90}| & Jones, Baker, and Williams (1990) \\
\verb|\citep*{jon90}| & (Jones, Baker, and Williams, 1990) \\
\verb|\citet{jon90,jam91}| & Jones et al. (1990); James et al. (1991) \\
\verb|\citep{jon90,jam91}| & (Jones et al., 1990; James et al. 1991) \\
\verb|\citep{jon90,jon91}| & (Jones et al., 1990, 1991) \\
\verb|\citep{jon90a,jon90b}| & (Jones et al., 1990a,b) \\
\end{tabular}

Alternatively, use {\tt\string\citealt} and {\tt\string\citealp} to produce
a cite without parentheses:

\begin{tabular}{ll}
\textbf{Command} & \textbf{Output} \\
\verb|\citealt{jon90}| & Jones et al. 1990 \\
\verb|\citealt*\{jon90}| & Jones et al., 1990 \\
\verb|\citealp*\{jon90}| & Jones, Baker, and Williams, 1990 \\
\verb|\citealp{jon90,jam91}| & Jones et al., 1990; James et al., 1991 \\
\verb|\citealp[pg.~32]{jon90}| & Jones et al., 1990, pg.~32 \\
\end{tabular}

\clearpage

Finally, the {\tt\string\citeauthor} and {\tt\string\citeyear} commands can
be used to print only the author or year portion of the citation:

\begin{tabular}{ll}
\textbf{Command} & \textbf{Output} \\
\verb|\citeauthor{jon90}| & Jones et al. \\
\verb|\citeauthor*{jon90}| & Jones, Baker, and Williams \\
\verb|\citeyear{jon90}| & 1990 \\
\verb|\citeyearpar{jon90}| & (1990) \\
\end{tabular}

\end{document}
